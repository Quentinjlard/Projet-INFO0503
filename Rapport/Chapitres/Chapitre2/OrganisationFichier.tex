Dans notre archive, une fois décompressé, vous trouverez de nombreux dossiers et fichiers.
\begin{itemize}
    \item Le dossier ".vscode" si vous êtes sous windows et que vous utilisez le logiciel Visual Studio Code comme nous, ce dossier apparaîtra. Il contient le fichier "Launch.json" , et nous reviendrons dessus dans la sous-partie "Compilation Windows".
    \item   Le dossier "cls", vous retrouverez l'ensemble des ".class" issue de la compilation JAVA.
    \item   Le dossier "doc" regroupe l'ensemble de la documentation de notre projet. Ce dossier n'est pas réellement utile car nous n'avons pas réalisé de documentation présise à l'interieur de notre projet par manque de temps. Nous avons commenté directement dans notre code sans utiliser les options et la syntaxe de la java doc.
    \item   Le dossier "lib" contient les librairies qui sont nécessaires pour notre projet. Dans ce projet, nous avons uniquement utilisé la librairie JSON, intitulé ici "json-20220924.jar".
    \item Le dossier "Rapport" contient l'ensemble des fichiers ".tex" du rapport que vous êtes en train de lire, car, en effet le rapport à été realisé sous LaTEX.
    \item Le dossier "src" contient l'ensemble des fichier sources ".java". Dans ce dossier, il existe un sous-dossier pour chaque entité présente dans notre modélisation qui seront exposés dans le Chapitre 4. Il y a également la présence d'un fichier "Lanceur.java" celui-ci sert à lancer l'ensemble des serveur java en multi-threadé en un seul lancement pour plus de simplicité. 
    \item De plus, dans le dossier "src", il existe le dossier "revendeur", le contenu de celui-ci doit être mis dans un serveur WEB PHP comme par exemple wamp ou autre.
    \item Le fichier "commande.txt" regroupe les commandes utiles pour la compilation et le lancement du projet.
     \item Le fichier "config.json" regroupe l'ensemble des ports et adresse des différents serveurs, il seront utilisés par le lanceur pour créer les différentes entités.

\end{itemize}

\newpage